\section{Introducing the Topic}

\subsection{A Piece of History}
\paragraph{}
In 1989, a Japanese company specialised in the video game industry called Sega Enterprise released a gaming platform under the name of Sega Genesis. For a game to be released on the console, it had to either be developed by third party developers who had agreed on their licensing deal or by a subsidiary company of Sega. It was mandatory for the licensees to pay an extra \$10 to \$15 per cartridge\footnote{Cartridges are removable enclosures that contain video games.} over the hardware manufacturing costs and to make the licensor, Sega, their exclusive publisher~\cite{Kent:2001:UHV:559522}. As a result, it would have prevented third party developers from making games not designed for the Genesis. To enforce their business plans, Sega implemented a protection mechanism inside the console for it to reject unofficial cartridges, which was kept secret from the outside world.

\paragraph{}
During that era, another video game company named Accolade took the decision to port their PC games to the Genesis, but without agreeing upon the licensing deal. The main obstacle was for them to find a way to bypass the protection mechanism embedded in the console to allow their cartridges to be accepted as legitimate ones. Using specific analysis tools, the company successfully understood the inner working of the console and defeated the protection mechanism, which gave them the necessary knowledge to port their games without the authorization of Sega.

\paragraph{}
Thereafter, Sega sued Accolade for copyright infringement as the tools used to extract the knowledge out of the Genesis had to generate intermediate copies of what is contained inside the console's memory. Accolade initially lost the lawsuit but appealed the verdict and the court, in the end, ruled these copies as fair use since they were not present in the final products, the cartridges. 

\paragraph{}
This story illustrates aptly the subject of this work as well as its implications. Should the reader be interested in a broader view of the story, he or she could investigate the following books: \textit{The Ultimate History of Video Games}~\cite{Kent:2001:UHV:559522} and \textit{Legal Battles that Shaped the Computer Industry}~\cite{graham1999legal}.

\subsection{Definition}
\begin{framed}
	\begin{definition} 
		\underline{Digital reverse engineering} is the process of extracting know-how or knowledge from a digital artefact.
	\end{definition}
\end{framed}

\paragraph{}
The above definition of digital reverse engineering has been inspired by the one presented in the introduction. One can notice the disappearance of the human-made condition as computer systems are de facto human inventions, and the appearance of the word \textit{digital}, which means that these artefacts are expressed by means of sequences of zeros and ones. A \textit{digital artefact} can more concretely be anything that lives inside a computer's memory, such as this work being in a pdf format, a network protocol, a program\footnote{A program is a static sequence of instructions that represent a computation~\cite{russinovich2012windows} or a executing computational process according to the context.}, an executable file\footnote{An executable (file) is a file that embodies the program in a way that makes it understandable for computers.}, or even a process\footnote{A process is a container for a set of resources used when executing the instance of the program (namely, the executable). It contains, amongst other things, the description of the computation to be performed~\cite{russinovich2012windows}.}.


